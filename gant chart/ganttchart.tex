\documentclass[landscape,12pt]{article}
  
% style
\usepackage[left=2.5cm,top=2cm,right=2.5cm,bottom=2cm,a4paper]{geometry}
\usepackage{fancyhdr}
\pagestyle{fancy}
\lhead{P\&O 3: Lichaams positie bepaling}
\rhead{Gantt-grafiek}
\renewcommand{\headrulewidth}{0.4pt}
\usepackage{color}

% gantt
\usepackage{pgfgantt}
\def\pgfcalendarweekdayletter#1{%
	\ifcase#1M\or T\or W\or T\or F\or S\or S\fi%
}

\definecolor{foobarblue}{RGB}{0,153,255}
\definecolor{foobaryellow}{RGB}{234,187,0}
\definecolor{grey}{RGB}{170,170,170}

\newganttchartelement{foobar}{
	foobar/.style={
		shape= rectangle,
		inner sep=0pt,
		draw=grey!70!blue,
		thick,
		fill=white,
		inner color=blue!10, outer color=blue!40, opacity=0.95
	},
	foobar incomplete/.style={
		/pgfgantt/foobar,
		draw=foobaryellow,
		bottom color=foobaryellow!50
	},
	foobar label font=\slshape,
	foobar left shift=0,%.1,
	foobar right shift=0%-.1
}

\begin{document}

% Een goede Gantt-chart maakt gebruik van \emph{links} en \emph{milestones}. Deze laatste definieer je in een lijstje onder de Gantt chart.

	\begin{figure}
		\centering
		\begin{ganttchart}[hgrid, vgrid, x unit=5mm, time slot format=isodate]{2020-10-01}{2020-11-08}
			\gantttitlecalendar{week,month=shortname,day,weekday=letter}
			\\
			\ganttgroup{Voorbereiding}{2020-10-01}{2020-10-8}\\
			\ganttfoobar[name = 1]{Openpose installatie}{2020-10-01}{2020-10-8}\\
			\ganttfoobar[name = 2]{Brainstom}{2020-10-01}{2020-10-8}\\
			\ganttfoobar[name = 3]{Openposetesten}{2020-10-08}{2020-10-15}\\
			\ganttlink{1}{3}
			\ganttfoobar[name = 4]{Literatuurstudie}{2020-10-08}{2020-10-22}\\
			\ganttgroup{Onderzoek}{2020-10-15}{2020-11-8}\\
			\ganttfoobar[name = 5]{0nderzoek basistoepassing}{2020-10-15}{2020-10-27}\\
			\ganttfoobar[name = 6]{Schrijven tussentijds verslag}{2020-10-15}{2020-10-30}\\
			\ganttfoobar[name = 7]{Uitgebreide toepassing}{2020-10-29}{2020-11-08}
			
		\end{ganttchart}	
	
	\end{figure}
	\begin{figure}
		\centering
		\begin{ganttchart}[hgrid, vgrid, x unit=5mm, time slot format=isodate]{2020-11-09}{2020-12-20}
			\gantttitlecalendar{week=7,month=shortname,day,weekday=letter} \\
			\ganttgroup{Onderzoek}{2020-11-9}{2020-12-10}\\
			\ganttfoobar[name=18]{Uitgebreide toepassing}{2020-11-09}{2020-12-10}\\
			\ganttgroup{Rapporteren}{2020-11-26}{2020-12-17}\\
			\ganttfoobar[name=19]{Schrijven eindverslag}{2020-11-26}{2020-12-15}\\
			\ganttfoobar[name=20]{Eindpresentatie voorbereiden}{2020-12-03}{2020-12-17}
			\ganttvrule[vrule/.append style={red, line width = 2pt}]{eindpresentatie}{2020-12-17}
			
		\end{ganttchart}
		

	\end{figure}
%\ganttgroup{voorbereiding}{2020-02-13}{2020-02-18} \\
%\ganttfoobar[name=1]{TO 1}{2020-02-13}{2020-02-18} \\
%\ganttfoobar[name=2]{indienen}{2020-02-18}{2020-02-18} \\
%\ganttlink{1}{2} % 21 must be completed before 22
%\ganttgroup{ontwerpen}{2020-02-13}{2020-03-13}\\
%\ganttfoobar[name=3]{ontwerp}{2020-02-13}{2020-02-28} \\
%\ganttfoobar[name=4]{bestelling}{2020-02-28}{2020-02-28} \\
%\ganttfoobar[name=5]{Solid edge model}{2020-02-28}{2020-03-06} \\	
%\ganttfoobar[name=6]{CAD model}{2020-03-06}{2020-03-13}\\					
%\ganttfoobar[name=7]{prototype}{2020-03-20}{2020-03-20}\\
%\ganttgroup{prototyping en testen}{2020-03-20}{2020-03-22}\\			
%\ganttlink{3}{4}
%\ganttlink{4}{7}
%\ganttlink{5}{7}
%\ganttlink{6}{7}
%\ganttfoobar[name=8]{programmeren}{2020-02-28}{2020-03-20}\\
%\ganttfoobar[name=9]{testen}{2020-03-20}{2020-03-22}\\
%\ganttmilestone{Mijlpaal I}{2020-03-05} % milestone completed	



\end{document}
