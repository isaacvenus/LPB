\documentclass{article}

\usepackage{graphicx, titling}

\title{Human Pose Estimation Toepassing
                                              }
\author{Isaac Venus\\
Stan Vanhecke\\
Mathieu Vanooteghem}
\date{30/10/2020}

\begin{document}
\pagestyle{empty}

\begin{center}
	{{\Large Subfaculteit wetenschappen}
	
	\vspace{1cm}
	
	\includegraphics[width=6cm]{2013-kulak-cmyk-highres.jpg}
	
	\vspace{1cm}
	
	\Large Probleemoplossen en ontwerpen 3}
	
	\vspace{2cm}

	{\Huge \textbf{Human Pose Estimation Toepassing}}
	
	\vspace{1cm}
	
	{\Large \textbf{Groepsnaam}}
	
	\vspace{1cm}
	
	{\Large \textbf{Isaac Venus}}\\
	{\Large \textbf{Stan Vanhecke}}\\
	{\Large \textbf{Mathieu Vanooteghem}}\\

\end{center}

\vspace{2cm}
{\Large Titularis : Koen Van Den Abeele}
\vspace{1cm}

{\Large Begeleider : Jens Goemaere}


\vspace{1cm}

\begin{center}
	{\Large Academiejaar 2020 - 2021}
\end{center}

\section*{Inleiding}
Aan materiaal in de medische wereld hangt steeds een stevig prijskaartje, daarom vroegen wij ons af of het niet mogelijk zou zijn om hulpmiddelen te ontwikkelen die heel budget vriendelijk zijn. Wat centraal staat is dat we hierbij geen gebruik maken van hoogtechnologische apparatuur en dat we ons beperken tot het gebruik van de camera's on onze laptops of gsm's. Het gaan dus meer om software dan het concreet bouwen van een apparaat. Hierbij richten wij ons op het gebied van 'Human Pose Estimation' (HPE), ook wel lichaamspositie bepaling. Dit is een techniek dat aan de hand van neurale netwerken en deep learning de positie van personen schat op een foto.

\section{HPE, revolutionair?}
(uitleggen hoe het werkt enzo, uitleggen waarom het zo gemakkelijk en handig is?)

\section{Brainstorm}
Het doel is ergens in de paramedische wereld iets te vinden dat mogelijks gebruik van maken van HPE, waardoor een een goedkope oplossing zou ontwikkeld kunnen worden. Hierbij testen we dan mogelijkse valkuilen en analyseren of dit een goede en gemakkelijke oplossing zou zijn. We proberen dit dan allemaal op een methodische manier uit te werken, waarbij we van een specifiek geval steeds breder gaan kijken.
\section{Voorlopige resultaten}

\section{Voorlopige conclusies}

\section{Vakintegratie}
Ons project kan ook aan enkele vakken gelinkt worden die we in onze eerste 3 semesters van de opleiding ingenieurswetenschappen hebben gehad. Het meest voor de hand liggende vak is natuurlijk Beginselen van Programmeren.
\section{Planning}
(gantt chart enzo\texttt{})

\section{Referenties}


\end{document}