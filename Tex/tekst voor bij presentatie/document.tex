\documentclass{article}

\title{Begeleidende tekst}
\date{\today}
\author{Barack Obama}

\begin{document}
	\maketitle
	\section{Openingsslide}
		Alle aanwezigen groeten en ons voorstellen en zeggen wat we gaan doen.
	\section{Overzicht}
		Kort overlopen wat we allemaal gaan zeggen.
	\section{Inleiding}
		Beetje preciezer zeggen wat we hebben gedaan en waarom we het gedaan hebben (stevig prijskaartje aan medische toepassingen)
		
		Dan onze oplossing (HPE) en waarmee we gaan werken om de kosten te drukken (alledaagse technologie)
		
		Verder heel kort zeggen wat onze toepassingen zijn, hier komen we natuurlijk later op terug.
	\section{HPE, revolutinair?}
		 Alles uitgebreid vertellen wat er op de slide staat.
	\section{Werking van Openpose}
		Idem
	\section{Bepalen van de hoek tussen arm en borst}
		Heel simpele eerste toepassing, eigenlijk gewoon om te leren programmeren met de python api van openpose. Maar we wilden toch al iets "nuttig" doen dus hebben we hiervoor gekozen.
	\section{Resultaten en conclusies}
		Een beetje praten over de positie van waaruit de foto moet worden getrokken
	\section{Wat is een bikefit?}
		Gewoon uitleg geven, alles staat er al
	\section{wijzigen zadelhoogte}
		vooral zadelhoogte beïnvloedt kniehoek (verticale beweging)\\
		de optimale hoek is tussen 140 en 145 graden maar met het programma kan je elke hoek kiezen\\
		het programma berekent de wijziging in pixels omdat het natuurlijk de lengtes niet kent\\
		dit kan je dan wijzigen door het een referentie te geven (bv lengte van je dijbeen) waaruit het dan van elk lichaamsdeel de lengte kan bepalen.
	\section{wijzigen stuurpenlengte}
		vooral stuurpenlengte beïnvloedt de schouderhoek (horizontale beweging)\\
		de optimale hoek is tussen de 90 en 95 graden\\
		slechts een beperkt aantal lengtes beschikbaar
	\section{Conclusies}
		invloed resolutie: verandering in resolutie geeft en te grote variatie op de schatting van openpose\\
		pose is slechts een schatting: natuurlijk kunnen er gewoon hierdoor al fouten in zitten\\
		fout in omzetting naar cm: er wordt een input van een persoon gevraagd, hier zitten er sowieso fouten in, al is het gewoon maar dat er per ongeluk te veel of te weinig gemeten wordt.
	\section{demonstratie}
		hier moet je demonstreren
	\section{Richtlijnen voor een goeie squat}
		uitleg geven bij de foto's
	\section{conclusies}
		minder precies, het moet gewoon een richtlijn geven van hoe goed je bezig bent. Het is dus beter geschikt voor openpose
	\section{demonstratie}
		hier moet je nogmaals demonstreren
	\section{besluit}
		medische toepassingen zijn hier ook de bikefitting (dat is redlijk medisch denk ik)\\
		ik wil er graag bij vermelden dat de installatie echt niet simpel was en dat er hier ook veel tijd is in gekropen om het bij iedereen te doen werken.
	\section{einde}
		Vragen?
\end{document}
