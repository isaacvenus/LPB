%% This classfile tries to implement the lay-out of the beamer style beamerthemekuleuven2. This .sty-file can be downloaded here: https://www.kuleuven.be/communicatie/marketing/templates/presentatiemateriaal/index.html Not all options of the style are implemented in this class, for its purpose is merely to mimic the lay-out, and to provide a way to change the lay-out of presentations that were made using the "old" kulakbeamer class. For new documents, we recommend using the .sty-file instead.

\documentclass
   [kulak] % options: kul or kulak (default), handout 
   {kulakbeamer}

\usepackage[dutch]{babel}
\usepackage[utf8]{inputenc}
\usepackage[T1]{fontenc}

\title[Korte titel]{Human Pose Estimation}
\subtitle{Toepassing}
\author[Korte naam]{Mathieu Vanooteghem, Isaac Venus en Stan vanhecke} 
\institute[Kulak]{KU Leuven Kulak}
\date{Academiejaar 2020 -- 2021}

%% Overview at begin of each section; delete if unwanted.

\AtBeginSection[]{
	\begin{frame}
	\frametitle{Overzicht} %Change to "Outline" for English presentation
	{
		\hypersetup{hidelinks} %disable link colors
		\hfill	{\large\parbox{.95\textwidth}{\tableofcontents[currentsection,hideothersubsections]}}
	}
\end{frame}}

\begin{document}

\begin{titleframe}
\titlepage
\end{titleframe}

\begin{outlineframe}[Overzicht]
\tableofcontents
\end{outlineframe}

 % % % Here you go  % % % 

\section{Inleiding}

\begin{frame}
\frametitle{Inleiding}
	\begin{itemize}
		\item stevig prijskaartje aan medische toepassingen
		\item werken met alledaagse technologie
		\item HPE: schatten van lichaamspositie
		\item verschillende toepassingen
	\end{itemize}
\end{frame}



\section[Korte titel]{Theoretische achtergrond}

\begin{frame}
\frametitle{HPE, revolutionair?}
\end{frame}

\begin{frame}
\frametitle{Werking van Openpose}
\end{frame}



\section{Toepassing 1: Opvolgen van revalidatie na schouderoperatie}

\begin{frame}
\frametitle{Bepalen van de hoek tussen arm en lichaam}
\end{frame}

\begin{frame}
\frametitle{Resultaten en conclusies}
\end{frame}



\section{Toepassing 2: Fietspositie bepalen}

\begin{frame}
	\frametitle{Wat is een bikefit?}
\end{frame}

\begin{frame}
	\frametitle{Algoritme voor het wijzigen van de zadelhoogte}
\end{frame}

\begin{frame}
	\frametitle{Algoritme voor het wijzigen van de stuurpenlengte}
\end{frame}

\begin{frame}
	\frametitle{Conclusies}
\end{frame}

\begin{frame}
	\frametitle{Demonstratie}
\end{frame}



\section{Toepassing 3: Correct uitvoeren van fitnessoefeningen}

\begin{frame}
	\frametitle{Richtlijnen voor een goeie squat}
\end{frame}

\begin{frame}
	\frametitle{Algoritme voor het bepalen van een goeie squat}
\end{frame}

\begin{frame}
	\frametitle{Conclusies}
\end{frame}

\begin{frame}
	\frametitle{Demonstratie}
\end{frame}



\section{Besluit}

\begin{frame}
\frametitle{Besluit}
Afsluitende tekst.
\end{frame}

\end{document}
